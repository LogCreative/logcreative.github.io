\documentclass[10pt,a4paper]{article}
% \pagestyle{plain}
\renewcommand{\tablename}{表}
\usepackage{amsmath, amssymb}
\usepackage{geometry}
\usepackage{graphicx}
\usepackage{float}
\usepackage{multicol}
\usepackage{multirow}
\usepackage{xcolor}
\usepackage[colorlinks]{hyperref}
\usepackage{caption}
\definecolor{grey}{rgb}{0.8,0.8,0.8}
\definecolor{darkgreen}{rgb}{0,0.3,0}
\definecolor{darkblue}{rgb}{0,0,0.3}
\usepackage{enumitem}
\usepackage{tcolorbox}
\tcbuselibrary{skins,breakable,raster,listingsutf8}
\lstset{language=[x86masm]Assembler}
\newtcbinputlisting{code}[3]{colframe=yellow!30!black,listing options={showstringspaces=false,
		showspaces=false,%
		tabsize=2,%
		basicstyle={\ttfamily\normalsize},%
		keywordstyle=\color{blue!80!black}\bfseries,%
		identifierstyle=,%
		commentstyle=\color{green!50!blue}\itshape,%
		stringstyle=\color{green!50!black},%
		rulesepcolor=\color{gray!20!white},
		breaklines,
		columns=flexible,
		extendedchars=false},listing file={code/#1},title={\textbf{#2}\\\small #3},listing only, breakable}
\usepackage{array}
\tcbset{fonttitle=\bfseries,breakable,
	colback=white,
	every box/.style={enhanced,
		before=\par\smallskip,after=\par\smallskip},
	every box on layer 2/.style={reset,every box,colback=yellow!10!white,
		drop fuzzy shadow}}
\geometry{left=1cm,right=1cm,top=1cm,bottom=1.5cm}
\graphicspath{{img/}}
%\def\img#1#2{\begin{figure}[htb]\centering\includegraphics[width=0.7\columnwidth]{img/#1.png}\caption{#2}\label{fig:#1}\end{figure}}
\def\img#1#2{\begin{tcolorbox}[title=#2]\includegraphics[width=\textwidth]{#1.png}\end{tcolorbox}}
\def\pdf#1#2{\begin{tcolorbox}[title=#2]\includegraphics[width=\textwidth]{#1.pdf}\end{tcolorbox}}
\setlist{nosep}
\setlength{\parindent}{0pt} 
\begin{document}
    \title{Computer Graphics}
    \date{}

    \begin{multicols}{3}
        \maketitle

        % \input{chapter/pipeline.tex}

		\section{Pipeline}

		\begin{itemize}
			\item[IN] Application
			\item Vertex Processing
			\item Triangle Processing
			\item Rasteriztion
			\item Fragment Processing
			\item Framebuffer Operations
			\item[OUT] Display
		\end{itemize}

		\section{Transformation}

		Translation
		\begin{equation*}
			\mathbf{T}(t_x,t_y,t_z) = \begin{pmatrix}
				1 & 0 & 0 & t_x \\
				0 & 1 & 0 & t_y \\
				0 & 0 & 1 & t_z \\
				0 & 0 & 0 & 1 \\
			\end{pmatrix}
		\end{equation*}

		Rotation
		\begin{equation*}
			\mathbf{R}(\phi) = \begin{pmatrix}
				\cos\phi & -\sin\phi \\
				\sin\phi & \cos\phi \\
			\end{pmatrix}
		\end{equation*}

		Scale
		\begin{equation*}
			\mathbf{S}(s_x,s_y,s_z) = \begin{pmatrix}
				s_x & 0 & 0 & 0 \\
				0 & s_y & 0 & 0 \\
				0 & 0 & s_z & 0 \\
				0 & 0 & 0 & 1 \\
			\end{pmatrix}
		\end{equation*}

		Shear
		\begin{equation*}
			\mathbf{H}_{xz}(s) = \begin{pmatrix}
				1 & 0 & s & 0 \\
				0 & 1 & 0 & 0 \\
				0 & 0 & 1 & 0 \\
				0 & 0 & 0 & 0 \\
			\end{pmatrix}
		\end{equation*}

		General Transformation (\textsc{TRaSh})
		\begin{equation*}
			\mathbf{TRSO}
		\end{equation*}

		Point
		\begin{equation*}
			(x,y,z,1)^T
		\end{equation*}

		\noindent Normalized Device Coordinates (NDC)
		\begin{equation*}
			(x,y,z,w)^T = (\frac{x}{w},\frac{y}{w},\frac{z}{w})^T
		\end{equation*}

		Vector
		\begin{equation*}
			(x,y,z,0)^T
		\end{equation*}

		Euler Angles (roll, pitch, yaw)
		\begin{equation*}
			\mathbf{R}_{xyz}(\alpha,\beta,\gamma) = \mathbf{R}_x(\alpha)\mathbf{R}_y(\beta)\mathbf{R}_z(\gamma)
		\end{equation*}

		Rodrigues' Rotation Formula: Rotation by angle $\alpha$ around axis $\mathbf{n}$.
		\begin{equation*}
			\begin{split}
				\mathbf{R}(\mathbf{n},&\alpha)\\ =& \cos(\alpha)\mathbf{I} \\&+ (1-\cos(\alpha))\mathbf{n}\mathbf{n}^T \\&+ \sin(\alpha)\begin{pmatrix}
					0 & -n_z & n_y \\
					n_z & 0 & -n_x \\
					-n_y & n_x & 0 \\
				\end{pmatrix}
			\end{split}
		\end{equation*}

		View Transformation (Book p. 67)
		\begin{align*}
			M_\text{view} &= R_\text{view}T_\text{view}\\
			T_\text{view} &= \begin{pmatrix}
				1 & 0 & 0 & -x_e \\
				0 & 1 & 0 & -y_e \\
				0 & 0 & 1 & -z_e \\
				0 & 0 & 0 & 1 \\
			\end{pmatrix}\\
			R_\text{view} &= \begin{pmatrix}
				x_{\hat{g}\times\hat{t}} & y_{\hat{g}\times\hat{t}} & z_{\hat{g}\times\hat{t}} & 0 \\
				x_t & y_t & z_t & 0 \\
				x_{-g} & y_{-g} & z_{-g} & 0 \\
				0 & 0 & 0 & 1 \\
			\end{pmatrix}
		\end{align*}

		Orthographic Projection (Book p. 94)
		\begin{align*}
			&\mathbf{P}_o = \mathbf{S}(\mathbf{s})\mathbf{T}(\mathbf{t}) \\
				&= \begin{pmatrix}
					\frac{2}{r-l} & 0 & 0 & 0 \\
					0 & \frac{2}{t-b} & 0 & 0 \\
					0 & 0 & \frac{2}{n-f} & 0 \\
					0 & 0 & 0 & 1 \\
				\end{pmatrix}\\&\begin{pmatrix}
					1 & 0 & 0 & -\frac{r+l}{2} \\
					0 & 1 & 0 & -\frac{t_b}{2} \\
					0 & 0 & 1 & -\frac{n+f}{2} \\
					0 & 0 & 0 & 1 \\
				\end{pmatrix} \\
				&= \begin{pmatrix}
					\frac{2}{r-l} & 0 & 0 & -\frac{r+l}{r-l} \\
					0 & \frac{2}{t-b} & 0 & -\frac{t+b}{t-b} \\
					0 & 0 & \frac{2}{f-n} & -\frac{f+n}{f-n} \\
					0 & 0 & 0 & 1 \\
				\end{pmatrix}
		\end{align*}

		Perspective Transform Matrix (Book p. 99)
		\begin{equation*}
			\mathbf{P}_p = \begin{pmatrix}
				\frac{n}{r} & 0 & 0 & 0 \\
				0 & \frac{n}{t} & 0 & 0 \\
				0 & 0 & -\frac{f+n}{f-n} & -\frac{2fn}{f-n} \\
				0 & 0 & -1 & 0 \\
			\end{pmatrix}
		\end{equation*}

		Canonical Cube to Screen: transform $[-1,1]^2$ to $[0,w]\times[0,h]$.
		\begin{equation*}
			\mathbf{M}_\text{viewport} = \begin{pmatrix}
				\frac{w}{2} & 0 & 0 & \frac{w}{2} \\
				0 & \frac{h}{2} & 0 & \frac{h}{2} \\
				0 & 0 & 1 & 0 \\
				0 & 0 & 0 & 1 \\
			\end{pmatrix}
		\end{equation*}

		\section{Shading Basics}
		Gooch Shading Model (Book p. 104)
		\begin{align*}
			\mathbf{c}_\text{shaded} =& s\mathbf{c}_\text{hightlight} \\ 
			&+ (1-s)(t\mathbf{c}_\text{warm}+(1-t)\mathbf{c}_\text{cool})
		\end{align*}

		Transparency (\textbf{over} operator)
		\begin{equation*}
			\mathbf{c}_o = \alpha_s\mathbf{c}_s + (1-\alpha_s)\mathbf{c}_d
		\end{equation*}

		\section{Illumination}

		Diffuse Shading (\textbf{Lambertian})
		\begin{equation*}
			L_d = k_d\frac{I}{r^2}\max(0,\mathbf{n}\cdot\mathbf{l})
		\end{equation*}

		Specular Shading (\textbf{Blinn-Phong})
		\begin{equation*}
			L_s = k_s\frac{I}{r^2}\max(0,\mathbf{n}\cdot\mathbf{h})^p
		\end{equation*}
		where $\mathbf{h} = \text{bisector}(\mathbf{v},\mathbf{l})$ (see p. 336).

		Ambient Shading
		\begin{equation*}
			L_a = k_aI_a
		\end{equation*}

		Blinn-Phong Reflection Model
		\begin{align*}
			L &= L_a + L_d + L_s \\
			&= k_aI_a + k_d\frac{I}{r^2}\max(0,\mathbf{n}\cdot\mathbf{l})\\&+ k_s\frac{I}{r^2}\max(0,\mathbf{n}\cdot\mathbf{h})^p
		\end{align*}

		\begin{description}
			\item[Flat Shading] shade each triangle/face.
			\item[Gouraud Shading] shade each vertex.
			\item[Phong Shading] shade each pixel.  
		\end{description}

		MSAA p. 139

		\section{Texture}
		Corresponder function
		\begin{itemize}
			\item wrap
			\item mirror
			\item clamp
			\item border
		\end{itemize}

		Barycentric Coordinates
		\begin{equation*}
			(x,y) = \alpha A + \beta B + \gamma C
		\end{equation*}
		where $\alpha + \beta + \gamma = 1$.

		\begin{align*}
			\alpha &= \frac{S_{BxC}}{S} \\
			\beta &= \frac{S_{AxC}}{S} \\
			\gamma &= \frac{S_{AxB}}{S}
		\end{align*}

		\paragraph{Bump Mapping} Access a texture to modify the surface normal instead of using a texture to
		change a color component in the illumination equation. Store three vectors: vertex normal $\mathbf{n}$, tangent $\mathbf{t}$, and bitangent $\mathbf{b}$.

		\begin{equation*}
			\begin{pmatrix}
				t_x & t_y & t_z & 0 \\
				b_x & b_y & b_z & 0 \\
				n_x & n_y & n_z & 0 \\
				0 & 0 & 0 & 1 \\
			\end{pmatrix}
		\end{equation*}

		\emph{A problem with bump and normal mapping is that the bumps
		never shift location with the view angle, nor ever block each
		other. }

		\paragraph{Parallax Mapping} take an educated guess of what should be seen in a pixel by examining the
		height of what was found to be visible. The bumps are stored in a heightfield texture.

		\paragraph{Environment Map} A function from the sphere to colors, stored as a texture.

		\section{Geometry}

			Represent Geometry
		\begin{itemize}
			\item Implicit
			\begin{itemize}
				\item Level set
				\item Algebraic surface $f(\mathbf{p})<0$ Inside
				\item Distance functions
			\end{itemize}
			\item Explicit
			\begin{itemize}
				\item Point cloud
				\item Polygon mesh
				\item Subdivision, NURBS (p. 781)
			\end{itemize}
		\end{itemize}

		B\'{e}zier Curve (p. 720) 
		\begin{align*}
			\mathbf{b}_0^1(\mathbf{t})&=(1-t)\mathbf{b}_0 + t\mathbf{b}_1\\
			\mathbf{b}_0^2(\mathbf{t})&=(1-t)^2\mathbf{b}_0 + 2t(1-t)\mathbf{b}_1 + t^2 \mathbf{b}_2 \\
			&\cdots \\
			\mathbf{b}_0^n(t) &= \sum_{j=0}^nB_j^n(t) \mathbf{b}_j
		\end{align*}
		where Bernstein polynomials
		\begin{equation*}
			B_i^n(t) = \binom{n}{i}t^i(1-t)^{n-i}
		\end{equation*}

		de Casteljau Algorithm (p. 737)

		\section{Mesh}

		Local Mesh Operations
		\begin{description}
			\item[Edge Flip] Change the division method of triangles
			\item[Edge Split] Get more triangles
			\item[Edge Collapse] Replace edge with a single vertex
		\end{description}

		Global Mesh Operations
		\begin{description}
			\item[Mesh Subdivision] upsampling
			\item[Mesh Simplification] downsampling --- Quadric Error Metrics (p. 708)
			\item[Mesh Regularization] same number of triangles 
			\item[Loop Subdivision]  Split each triangle into four (p. 758)
			\item[Catmull-Clark Subdivision] Regular Quad Mesh (p. 762)
		\end{description}

		\section{Raytracing}

	 	\paragraph{Ray Casting}: Perform
		shading calculation
		here to computer color of pixel
		(e.g. Blinn Phong model)

		\paragraph{Recursive Ray Tracing (Whitted-Style)}
		\begin{enumerate}
			\item Trace secondary rays
			recursively until hit a non
			specular surface (or max
			desired levels of recursion)
			\item At each hit point, trace shadow
			rays to test light visibility (no
			contribution if blocked)
			\item Final pixel color is weighted
			sum of contributions along rays,
			as shown
			\item Gives more sophisticated
			effects (e.g. specular reflection,
			refraction, shadows), but we
			will go much
			further to derive a physically
			based illumination model
		\end{enumerate}

		Ray Equation (p. 943)
		\begin{equation*}
			\mathbf{r}(t) = \mathbf{o} + t\mathbf{d}
		\end{equation*}

		Plane Equation
		\begin{equation*}
			(\mathbf{p}-\mathbf{p^\prime})\cdot\mathbf{N}=0
		\end{equation*}

		AABB (Axis Aligned
		Bounding Box) p. 944

		OBB (Oriented Bouding Box) p.945

		Ray intersection with AABB
		\begin{align*}
			t_\text{enter} &= \max t_\text{min} \\
			t_\text{exit} &= \min t_\text{max} \\
			\begin{cases}
				t_\text{enter} < t_\text{exit} \\
				t_\text{exit} \geq 0
			\end{cases}
		\end{align*}

		Spatial Partitioning
		\begin{itemize}
			\item Oct-Tree p. 824
			\item KD-Tree p. 822
			\item BSP-Tree p. 823
		\end{itemize}

    \end{multicols}

\end{document}