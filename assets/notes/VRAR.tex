\documentclass[]{ctexart}
\usepackage[]{amsmath}
\usepackage[colorlinks]{hyperref}
% \usepackage[margin=1cm]{geometry}
\begin{document}
    \title{复习}
    \maketitle
        \section{光线追踪}
\paragraph{光线追踪基本原理和算法}



\paragraph{光线追踪加速结构}

\subparagraph{KD-Tree (K-dimensional Tree)}

每次切分都是按照某一轴切分,是一个二叉树。

\begin{enumerate}
    \item 如果中间节点 (internal node) 与光线相交,则和两个子节点都可能相交,子节点继续判断
    \item 如果叶子节点 (leaf node) 与光线相交,则叶子节点里的所有物体都要和光线做相交判断。
\end{enumerate}

non-overlapping

\subparagraph{BVH (Bouding Volume Hierarchy)} 

may be overlapping

\begin{enumerate}
    \item 如果光线和叶子节点相交,就和其中所有物体算一遍求交
    \item 如果光线和中间节点相交,则分别处理两个子节点,返回更近的一次相交
\end{enumerate}

\paragraph{BRDF (Bidirectional Reflectance Distribution Function)}

\begin{equation}    
    f_r(\omega_i\rightarrow w_r)=\frac{d L_r(\omega_r)}{d E_i(\omega_i)}=\frac{d L_r(\omega_r)}{L_i(\omega_i)\cos\theta_i d\omega_i}
\end{equation}

\begin{equation}
    L_r(\mathbf{p},\omega_r) = \int_{H^2} f_r(\mathbf{p},\omega_i\rightarrow\omega_r)L_i(\mathbf{p},\omega_i)\cos\theta_i d\omega_i
\end{equation}

\paragraph{渲染方程}

\begin{equation}
    \underbrace{L_o(\mathbf{p},\omega_o)}_{\text{outgoing radiance}}= \underbrace{L_e(\mathbf{p},\omega_o)}_{\text{emission}} + \int_{\Omega^+}\underbrace{f_r(\mathbf{p},\omega_i\rightarrow\omega_o)}_{\text{BRDF}}\underbrace{L_i(\mathbf{p},\omega_i)}_{\text{incident radiance}}\cos\theta_i d\omega_i
\end{equation}

\begin{equation}
    \underbrace{L_o(\mathbf{p},\omega_o)}_{\text{outgoing radiance}}=\int_{\Omega^+}\underbrace{f_r(\mathbf{p},\omega_i\rightarrow\omega_o)\cos\theta_i}_{\text{(cosine-weighted) BRDF}}\underbrace{L_i(\mathbf{p},\omega_i)}_{\text{incident lighting}} \underbrace{V(\mathbf{p},\omega_i)}_{\text{visibility}} d\omega_i \label{eq:rtr}
\end{equation}


\section{路径追踪}
\paragraph{路径追踪算法}
% n 注:不考蒙特卡洛积分

\begin{equation}
    L_o(p,\omega_o)\approx\frac{1}{N}\sum_{i=1}^N\frac{L_i(p,\omega_i)f_r(p,\omega_i,\omega_o)(n\cdot\omega_i)}{p(\omega_i)}
\end{equation}

\begin{verbatim}
shade(p, wo)
    Randomly choose N directions wi~pdf
    Lo=0.0
    For each wi
        Trace a ray r(p, wi)
        If ray r hit the light
            Lo += (1/N)*L_i*f_r*cosine/pdf(wi)
    Return Lo
\end{verbatim}

\section{实时阴影}

\eqref{eq:rtr} 可以被近似为

\begin{equation}
    L_o(\mathbf{p},\omega_o)\approx\frac{\int_{\Omega^+}V(\mathbf{p},\omega_i)d\omega_i}{\int_{\Omega^+}d\omega_i}\cdot\int_{\Omega^+}L_i(\mathbf{p},\omega_i)f_r(\mathbf{p},\omega_i,\omega_o)\cos\theta_i d\omega_i
\end{equation}

\paragraph{Shadow map}

\paragraph{PCF (Percentage Closer Filtering)}
% 注:不考VSSM,不考MIPMAP 和SAT,不考Moment Shadow Mapping,不考
% Distance Field 相关、Distance Field Soft Shadows

区域取平均。 

\paragraph{PCSS (Percentage Closer Soft Shadows)} 根据接受物到遮挡物的距离决定阴影的软硬(乘个比例)。

\begin{equation}
    w_\text{penumbra} = (d_\text{receiver} - d_\text{blocker}) \cdot w_\text{light} / d_\text{blocker}
\end{equation}

\begin{enumerate}
    \item Blocker search (getting the average blocker depth in a certain region)
    \item Penumbra estimation (use the average blocker depth to determine filter size)
    \item Percentage Closer Filtering
\end{enumerate}

\section{实时环境光}
\paragraph{Precomputed Radiance Transfer(PRT)}

\begin{equation}
    L(\mathbf{o}) = \int_\Omega L(\mathbf{i}) V(\mathbf{i})\rho(\mathbf{i},\mathbf{o})\max(0,\mathbf{n}\cdot\mathbf{i})d\mathbf{i}
\end{equation}

\subparagraph{diffuse}
\begin{equation}
    \begin{aligned}
        L(\mathbf{o}) &= \rho\int_\Omega L(\mathbf{i}) V(\mathbf{i})\max(0,\mathbf{n}\cdot\mathbf{i})d\mathbf{i} \\
        &\approx \rho\sum l_i\int_\Omega B_i(\mathbf{i})V(\mathbf{i})\max(0,\mathbf{n}\cdot\mathbf{i})d\mathbf{i} \\
        &\approx \rho\sum l_i T_i
    \end{aligned}
\end{equation}

% \begin{equation}
%     \begin{aligned}
%         L_o(\mathbf{p},\omega_o) = \int_{\Omega^+} L_i(\mathbf{p},\omega_i) f_r(\mathbf{p},\omega_i,\omega_o)\cos\theta_i V(\mathbf{p},\omega_i)d\omega_i
%     \end{aligned}
% \end{equation}

\subparagraph{glossy}
% n 注:不考Shading from Environment Lighting(split sum 方法)、Shadow from
% Environment Lighting。不考Spherical Harmonics 球谐函数基本概念、Prefiltered
% Environment Lighting和wavelet小波函数
\begin{equation}
    L(\mathbf{o})\approx \sum\left(\sum l_it_{ij}\right)B_j(\mathbf{o})
\end{equation}

\section{实时全局光}
\paragraph{RSM (Reflective Shadow Maps)}

公式 \ref{eq:rtr} 改写为
\begin{equation}
    L_o(p,\omega_o)=\int_{A_\text{patch}}L_i(\mathbf{q}\rightarrow\mathbf{p})V(\mathbf{p},\omega_i)f_r(\mathbf{p},\mathbf{q}\rightarrow\mathbf{p},\omega_o)\frac{\cos\theta_p\cos\theta_q}{\Vert q-p\Vert^2}dA        
\end{equation}

\begin{equation}
    E_p(x,n)=\Phi_p\frac{\max\{0,\langle n_p|x-x_p\rangle\}\max\{0,\langle n|x_p-x\rangle\}}{\lVert x-x_p\rVert^4}
\end{equation}

Depth, world coordinate, normal, flux

\paragraph{SSAO}
% n 注:不考LPV、VXGI,不考SSDO、SSR

每点发出光看比例。

\begin{equation}
    \begin{aligned}
        L_o^\text{indir}(\mathbf{p},\omega_o)&\approx\frac{\int_{\Omega^+}V(\mathbf{p},\omega_i)\cos\theta_i d\omega_i}{\int_{\Omega^+}\cos\theta_i d\omega_i}\cdot\int_{\Omega^+}L_i^\text{indir}(\mathbf{p},\omega_i)f_r(\mathbf{p},\omega_i,\omega_o)\cos\theta_i d\omega_i\\
        &=\frac{\int_{\Omega^+}V(\mathbf{p},\omega_i)\cos\theta_i d\omega_i}{\pi} L_i^\text{indir}(p)\cdot\frac{\rho}{\pi}\pi\\
        &=\underbrace{k_A}_\text{the weight-averaged visibility from all directions} \underbrace{L_i^\text{indir}(p)\rho}_\text{constant for AO}
    \end{aligned}
\end{equation}

\section{实时光线追踪}
\paragraph{Motion vector、Temporal filtering时序滤波}



\paragraph{联合双边滤波}
% n 注:不考Temporal Failure,不 考Large Filter、Outlier Removal ,不考SVGF、RAE
% • 非真实感渲染
% n 只考察任务的定义,不考具体的方法
% • 神经风格迁移
% n 只考察任务的定义,不考具体的方法
% • 内容生成
% n 只考察任务的定义,不考具体的方法
\end{document}