\documentclass[10pt]{article}
\usepackage{geometry}
\usepackage{CJKutf8}
\usepackage{multicol}
\usepackage{amsmath}
\usepackage{amssymb}
\usepackage{bm}
\usepackage{color}

\numberwithin{equation}{section}
\geometry{left=1.0cm,right=1.0cm,top=1.0cm,bottom=1.5cm}
\title{Thermal Physics Sheet}
\author{Log Creative}
\date{}
\begin{document}
\begin{multicols}{3}
\begin{CJK}{UTF8}{gbsn} % Windows 用户需要将字体改为 song
\maketitle
\section{压强}
状态方程

\begin{equation}
  pV=\nu RT
\end{equation}
微观公式
\begin{align}\nonumber
  p=nkT=\frac{2}{3}n\bar{\varepsilon}&=\frac{2}{3}n\left(\frac{1}{2}m\bar{v^2}\right) \\ &=\frac{2}{3}n\left(\frac{3}{2}m\bar{v_x^2}\right)
\end{align}
范德瓦尔斯
\begin{align}
  \left(p+\nu^2 \frac{a}{V_m^2}\right)&(V-\nu b)=\nu RT \\
  b&=4N_A \frac{4}{3}\pi \left( \frac{d}{2} \right)^2
\end{align}
Boltzmann
\begin{align}
p&=p_0 e^{-\frac{Mgz}{RT}}(\textrm{Const. }T) \\
n&=n_0 e^{-\frac{\varepsilon_0}{kT}}(\textrm{Const. }T)
\end{align}
\section{分量}

方均根(速率最大)
\begin{align}
  \sqrt{\bar{v^2}} &=\sqrt{\frac{3RT}{M}} \\
  \sqrt{\bar{v_x^2}} &=\sqrt{\frac{RT}{M}}
\end{align}
均值(速率中等)
\begin{align}
  \bar{v}= & \sqrt{\frac{8RT}{\pi M}} \\
  \bar{v_x}= & \sqrt{\frac{RT}{2\pi M}}
\end{align}
最可几速率(最概然速率,速率最小)
\begin{equation}
  v_p=\sqrt{\frac{2RT}{M}}
\end{equation}
(分量最概然速度大小为0.) \\
Maxwell
\begin{align}\nonumber
  \frac{\textrm{d}N_v}{N} &=\left(\frac{m}{2\pi kT}\right)^{\frac{3}{2}}e^{-\frac{mv^2}{2kT}}4\pi v^2 \textrm{d}v \nonumber \\ &=f(v)\textrm{d}v \\
  \frac{\textrm{d}N_{v_x}}{N} &=\left(\frac{m}{2\pi kT}\right)^{\frac{1}{2}}e^{-\frac{mv_x^2}{2kT}} \textrm{d}v \nonumber \\&=f(v_x)\textrm{d}v
\end{align}
碰撞
\begin{align}
N&=\frac{1}{4}n\bar{v}\\
\Delta N_{0\sim\beta v_{p(x)}}&=\frac{N}{2} \textrm{erf} \left(\beta\right)\\
\Delta N_{0\sim \beta v_{p}}&=N\left[ \textrm{erf} (\beta)-\frac{2}{\sqrt\pi}\beta e^{-\beta^2}\right] \\
\textrm{erf} \left(\beta\right)&=\frac{2}{\sqrt\pi}\int_{0}^{\beta}{e^{-x^2}\textrm{d}x}
\end{align}
能量按自由度均分定理
\begin{align}
  \frac{1}{2}m\bar{v^2}&=\frac{3}{2}kT \\
  \frac{1}{2}m\bar{v_x^2}&=\frac{1}{2}kT
\end{align}
\section{多项}
摩尔平均总能量(平动、转动、振动)
\begin{align}
  {\bar{\varepsilon}}_{\textrm{m}}&=\frac{1}{2}\left(t+r+2s\right)RT \nonumber \\ \nonumber
  &=\frac{3}{2}RT\textrm{(single)} \nonumber \\&=\frac{5}{2}RT\textrm{(hard dual)} \nonumber\\
  &=\frac{7}{2}RT\textrm{(elastic dual)} \nonumber \\ &=\frac{6}{2}RT\textrm{(multi)}
\end{align}
热容
\begin{align}
  C&=mc \\
  C_{\textrm{m}}&=Mc \\
  C_{V,\textrm{m}}&=\frac{1}{2}\left(t+r+2s\right)R
\end{align}
扩散
\begin{align}
  \sigma&=\pi d^2 \\
  \bar{Z}&=\sqrt2\sigma\bar{v}n \\
  \bar{\lambda}&=\frac{1}{\sqrt2\sigma n}
\end{align}
分子按自由程的分布:$N_0$中自由程大于$x$的分子数,求微分得连续分布
\begin{equation}
  N=N_0 e^{-\frac{x}{\bar{\lambda}}}
\end{equation}
粘滞、导热、扩散系数
\begin{align}
  \textrm{d}y&=-c\left(\frac{\textrm{d}x}{\textrm{d}z}\right)_{z_0}\textrm{d}S\textrm{d}t \\
  \eta&=\frac{1}{3}\rho\bar{v}\bar{\lambda} \\
  \kappa&=\eta c\\
  D&=\frac{\eta}{\rho} \\
  c&=\frac{C_{V,\textrm{m}}\nu}{m}
\end{align}

\section{热力学定律}
\paragraph{第一定律} 加进一个的系统中的热量+对系统所做的功=系统内能的增加
\begin{equation}
  Q+A=\Delta U
\end{equation}
\paragraph{第二定律}  不可能有这样一个过程,它的唯一结果\textbf{只是}从一个热库取出热量,并把它转化为功.\\
没有任何一台热机,在从$T_1$取得热量$Q_1$,而在$T_2$放出热量$Q_2$的过程中所做的功比可逆机更大.对于可逆机,
\begin{equation}
  W=Q_1-Q_2=Q_1\left(1-\frac{T_2}{T_1}\right)
\end{equation}
\paragraph{系统的熵}

如果$\Delta Q$是可逆地加在温度为$T$的系统中的热量,那么这个系统的熵增为
  \begin{equation}
  \Delta S=\frac{\Delta Q}{T}
  \end{equation}
  熵为:
  \begin{equation}
    S(\bm{V},\bm{T})=R\left(\ln \bm{V}+\frac{1}{\gamma-1}\ln \bm{T} \right)+a
  \end{equation}
  当$T=0$时,$S=0$(\textbf{第三定律}).此时的熵定义为:\\
  在\textbf{可逆变化}中,系统所有部分(包括热库)的总熵\textbf{不变}.\\
  在\textbf{不可逆变化}中,系统的总熵始终不断\textbf{增加}.\textbf{无摩擦的准静态过程}是可逆的.

\paragraph{通用公式}
\begin{align}
  Q_V=\Delta U&=\nu C_{V,\textrm{m}}\Delta T\\
  Q_p=\Delta H&=\nu C_{p,\textrm{m}}\Delta T \\
  A&=-\int_{V_1}^{V_2}p\textrm{d}V \\
  H& =U+pV \\
  T\textrm{d}S&=\textrm{d}U+p\textrm{d}V
\end{align}
\textbf{理想气体}
  \begin{align}
  \Delta U&=\nu C_{V,\textrm{m}}\Delta T \\
  C_{p,\textrm{m}}-C_{V,\textrm{m}}&=R \\
  \Delta H &= \nu C_{p,\textrm{m}}\Delta T
  \end{align}
\textbf{范德瓦尔斯气体}
  \begin{align}
  \Delta U&= \nu \left[ C_{V,\textrm{m}}\Delta T-a\Delta \left( \frac{1}{V_{\textrm{m}}} \right) \right]\\
  C_{p,\textrm{m}}-C_{V,\textrm{m}}&=\frac{R}{1-\frac{2a(V_{\textrm{m}}-b)^2}{RTV_{\textrm{m}}^3}}\\
  \Delta H_{\textrm{m}} &= (C_{V,m\textrm{}}+R)\Delta T -a \Delta \left( \frac{1}{V_{\textrm{m}}} \right)\nonumber \\&+\frac{RT_2b}{V_2-b}-\frac{RT_1b}{V_1-b}
  \end{align}

\section{热力学过程}
$A$:外界对系统所做的功\\
$Q$:系统从外界吸收的热量\\
$C_m$:摩尔热容\\
$\Delta S$:熵变\\
(理想气体)
\begin{center}
    \fbox{等容过程 $V=$Const.}
\end{center}
\begin{align}
  A&=0 \\
  Q&=\nu C_{V,\textrm{m}}(T_2-T_1) \\
  C_{V,\textrm{m}}&=\frac{R}{\gamma-1} \\
  \Delta S&=\nu C_{V,\textrm{m}}\ln\frac{T_2}{T_1}
\end{align}
\begin{center}
    \fbox{等压过程 $p=$Const.}
\end{center}
\begin{align}
  A&=-p(V_{2}-V_{1})\nonumber \\ &=-\nu R(T_2-T_1) \\
  Q&=\nu C_{p,\textrm{m}}(T_2-T_1) \\
  C_{V,\textrm{m}}&=\frac{\gamma R}{\gamma-1} \\
  \Delta S&=\nu C_{p,\textrm{m}}\ln\frac{T_2}{T_1}
\end{align}
\begin{center}
    \fbox{等温过程 $T=$Const.}
\end{center}
\begin{align}
  A&=-p_1V_{1}\ln\frac{V_{2}}{V_{1}} \nonumber \\&=-\nu RT_1\ln\frac{V_{2}}{V_{1}} \\
  Q&=-A \\
  C_{\textrm{m}}&=\infty \\
  \Delta S&=\nu R\ln\frac{V_2}{V_1}
\end{align}
\begin{center}
    \fbox{绝热过程 $Q=0$}
\end{center}
泊松方程
\begin{align}
  pV^{\gamma}&=\textrm{Const.} \\
  TV^{\gamma-1}&=\textrm{Const.} \\
  \frac{p^{\gamma -1}}{T^{\gamma}}&=\textrm{Const.}
\end{align}
\begin{align}
  A&= \frac{p_1V_{1}}{\gamma -1}\left[\left(\frac{V_{1}}{V_{2}}\right)^{\gamma -1}-1\right]\nonumber\\ &=\nu C_{V,\textrm{m}}(T_2-T_1) \\
  Q&=0 \\
  C_m&=0 \\
  \Delta S&=0
\end{align}
\begin{center}
    \fbox{多方过程 $pV^n=$Const.}
\end{center}

\begin{align}
  A&=\frac{p_1V_{1}}{n -1}\left[\left(\frac{V_{1}}{V_{2}}\right)^{n -1}-1\right]\nonumber\\ &=\frac{\nu R}{n-1}(T_2-T_1) \\
  Q&=\nu \left( C_{V,\textrm{m}}-\frac{R}{n-1} \right) (T_2-T_1)\\
  C_m&=\frac{\gamma -n}{1-n}C_{V,\textrm{m}} \\
  \Delta S&=\nu C_{V,\textrm{m}} (\gamma -n)\ln \frac{V_{2}}{V_{1}}
\end{align}
\begin{center}
    \fbox{自由膨胀 $A=0$}
\end{center}
这一过程不是准静态过程.
\begin{center}
    \fbox{绝热节流 $H=$Const.}
\end{center}

焦汤系数
\begin{equation}
  \alpha \equiv \lim_{\Delta p \rightarrow 0} \left( \frac{\Delta T}{\Delta p} \right)_H = \left(\frac{\partial T}{\partial p} \right)_H
\end{equation}




\begin{center}
\begin{tabular}{ccc}
  \hline
  $\alpha$ & 类型(室温下)  & 效应\\
  \hline
  % after \\: \hline or \cline{col1-col2} \cline{col3-col4} ...
  $+$ & 氮、氧、空气 & 制冷效应、正效应\\
  $-$ & 氢气 & 制温效应、负效应 \\
  0 & 理想气体 & 零效应$^{*}$ \\
  \hline

\end{tabular}
\begin{flushleft}
\small $^{*}$ 非理想气体的对应温度为\textbf{转换温度}.上转换温度$T^{\circ}=\frac{2a}{Rb}$.
\end{flushleft}
\end{center}



\section{其他热力学方程}
效率、冷却系数
\begin{align}
  \eta&=\frac{A}{Q_1}=1-\frac{|Q_2|}{Q_1} \\
  \varepsilon &=\frac{Q_2}{Q_1-Q_2}
\end{align}
卡诺、可逆
\begin{align}
  \eta&=1-\frac{T_2}{T_1} \\
  \varepsilon &=\frac{T_2}{T_1-T_2} \\
  \oint_{\textrm{invertible cycle}}\frac{\textrm{d}Q}{T}&=0
\end{align}
热力学定律常用(以及 4.9)
\begin{align}
  \left(\frac{\partial U}{\partial V}\right)_V&=T\left(\frac{\partial p}{\partial T}\right)_V-p \\
  \textrm{d}U&=\left(\frac{\partial U}{\partial T}\right)_V \textrm{d}T+\left(\frac{\partial U}{\partial V}\right)_T \textrm{d}U
\end{align}
熵
\begin{align}
  W&=\frac{N!}{\Pi_{i=1}^n N_i!} \\
  S&=k\ln W
\end{align}
大气温度梯度
\begin{align}
  \frac{\textrm{d}T}{\textrm{d}z}&=-\frac{\gamma-1}{\gamma} \frac{T}{p} \rho g\\
  \textrm{d}p&=-\frac{Mgp}{RT}\textrm{d}z(\textrm{Boltzmann})\\
  h&=\frac{C_{p,\textrm{m}}T_0}{Mg}\left[1-\left(\frac{p}{p_0}\right)^{\frac{\gamma -1}{\gamma}}\right]
\end{align}
\section{相变}
克拉珀龙方程衍生式
\begin{align}
  \frac{\textrm{d}p}{\textrm{d}T}&=\frac{l}{T(v_2-v_1)} \nonumber\\
  &=\frac{u_2-u_1+p(v_2-v_1)}{T(v_2-v_1)} \nonumber \\
  &\approx \frac{l}{Tv_2} \nonumber \\
  &\approx \frac{\Delta p}{\Delta T}
\end{align}
理想气体蒸气压方程
\begin{equation}
  \ln p=-\frac{L}{RT}+\textrm{Const.}
\end{equation}
范德瓦尔斯相变临界点参量
\begin{equation}
  \left\{
    \begin{aligned}
    T_k&=\frac{8a}{27bR}\\
    V_{\textrm{m}k}&=3b \\
    p_{k}&=\frac{a}{27b^2}
    \end{aligned}
  \right.
\end{equation}
\begin{equation}
  \frac{RT_k}{p_kV_{\textrm{m}k}}=\frac{8}{3}
\end{equation}
\section{常数}
\begin{equation}
  R=\frac{k}{N_A}
\end{equation}
\begin{center}
\begin{tabular}{cc}
  \hline
   & Value \\
  \hline
  % after \\: \hline or \cline{col1-col2} \cline{col3-col4} ...
  $R$ & 8.314\\
  $k$ & $1.381\times 10^{-23}$\\
  $N_A$ & $6.022\times 10^{23}$ \\
  \hline
\end{tabular}
\end{center}
\begin{equation}
  n=\frac{N_A}{V_{\textrm{m}}}
\end{equation}
\begin{center}
\begin{tabular}{cc}
  \hline
   & Value \\
  \hline
  % after \\: \hline or \cline{col1-col2} \cline{col3-col4} ...
  $V_{\textrm{m}}$(STP) & 22.41\\
  \hline
\end{tabular}
\end{center}
\begin{equation}
  A=eU
\end{equation}
\begin{center}
\begin{tabular}{cc}
  \hline
   & Value \\
  \hline
  % after \\: \hline or \cline{col1-col2} \cline{col3-col4} ...
  $e$ & $1.602\times 10^{-19}$C\\
  $e$V & $1.602\times 10^{-19}$J\\
  \hline
\end{tabular}
\end{center}
\end{CJK}
\end{multicols}
\end{document} 