\documentclass[12pt,a4paper]{article}
\usepackage{CJKutf8}
\usepackage[english]{babel}
%\usepackage[utf8x]{inputenc}
\usepackage{amsmath}
\usepackage{amssymb}
\usepackage{dsfont}
\usepackage{graphicx}
%\usepackage[UTF8]{ctex}
\usepackage{geometry}
\usepackage{bm}
\geometry{left=2.0cm,right=2.0cm,top=2.5cm,bottom=2.5cm}
\numberwithin{equation}{section}

\begin{document}
\begin{CJK}{UTF8}{gbsn} % Windows 用户需要将字体改为 song

\title{Mechanics Summary}
\author{Log Creative}
\date{}
\maketitle

\section{只涉及三维坐标的质点运动学}
\paragraph{自然坐标系}

\begin{equation}
  x(t)-x_0=\int_0^t \left( v_0 + \int_0^t a dt \right)dt
\end{equation}

\begin{equation}
  \bm{r}(t)-\bm{r}_0=\int_0^t \left( \bm{v}_0 + \int_0^t \bm{a} dt \right)dt
\end{equation}

\begin{equation}
  \bm{v}=\bm{\omega}\times \bm{r}
\end{equation}

\begin{equation}
  \bm{a}=\bm{a}_t+\bm{a}_n=\frac{dv}{dt}\bm{e}_t+\frac{v^2}{\rho} \bm{e}_n
\end{equation}

\paragraph{极坐标系}

\begin{align}
  \dot{\bm{e_r}}&=\dot{\theta}\bm{e_{\theta}} \\
  \dot{\bm{e_{\theta}}}&=- \dot{\theta}\bm{e_r} \\
  \bm{v}&=\dot{r} \bm{e}_r +r \dot{\theta} \bm{e}_{\theta} \\
  \bm{a}&=\left(\ddot{r}-r\dot{\theta}^2\right)\bm{e}_r+\left( r \ddot{\theta}+2 \dot{r} \dot{\theta} \right)\bm{e}_{\theta}
\end{align}

\paragraph{相对运动(伽利略变换)}
\begin{align}
  \bm{v} & =\bm{v}^{\prime}+\bm{v}_f \\
  \bm{a} & =\bm{a}^{\prime}+\bm{a}_f
\end{align}

\paragraph{匀速转动}
\begin{align}
  \bm{v} & = \bm{v}^{\prime} + \bm{\omega} \times \bm{r} \\
  \bm{a} & =\bm{a}^{\prime} + \bm{\omega} \times (\bm{\omega} \times \bm{r}^{\prime} )+2 \bm{\omega} \times \bm{v}^{\prime}
\end{align}

\section{加入了力的牛顿运动定律}
\paragraph{牛顿第二定律}
\begin{equation}
  \bm{F}=m\bm{a}
\end{equation}

\paragraph{虚拟力}
\begin{equation}
  \bm{F}_i=m\bm{a'}-m\bm{a}
\end{equation}

\paragraph{科里奥利力}

\begin{equation}
 m \bm{a}^{\prime}  =m \bm{a} - m \bm{\omega} \times (\bm{\omega} \times \bm{r}^{\prime} )+ 2 m \bm{v}^{\prime} \times  \bm{\omega}
\end{equation}

\section{更广泛适用的动量}
\paragraph{质点系动量定理}
\begin{equation}
  \bm{F}_{ex}=\frac{d\bm{p}}{dt}
\end{equation}

\paragraph{质心}
\begin{equation}
  \bm{F}_{ex}=m\bm{a}_c
\end{equation}
\begin{equation}
  \bm{r}_C=\frac{\int \bm{r} dm}{\int dm}
\end{equation}

\section{具有守恒性质的衡量 功与能}

\paragraph{引入定义}
\begin{align}
\bm{F}&=m\bm{C} \\
\bm{F}&=-\bm{\nabla}U \\
\bm{C}&=-\bm{\nabla} \Psi
\end{align}

\begin{equation}
  \bm{A}\cdot d\bm{A}=AdA
\end{equation}

\paragraph{质点系中功能原理}
\begin{align}
  W_{ex}+W_{ic}+W_{in}&=E_k(b)-E_k(a)\\
  W_{ex}+W_{in}&=E(b)-E(a)
\end{align}

\begin{align}
  E_k &= \frac{1}{2}mv_C^2 +E_{kC} = \frac{1}{2}mv_C^2 + \sum_i \frac{1}{2} m_i v_i^{\prime2} \\
  W_{ex}^{\prime}+W_{in}^{\prime} &=E^{\prime}-E_0^{\prime}
\end{align}

\paragraph{碰撞}
\begin{equation}
  e=\frac{v_2-v_1}{u_1-u_2}\in[0,1]
\end{equation}
\begin{align}
  v_1 &= \frac{m_1-em_2}{m_1+m_2}u_1+\frac{(1+e)m_2}{m_1+m_2}u_2 \\
  v_2 &= \frac{(1+e)m_1}{m_1+m_2}u_1-\frac{em_1-m_2}{m_1+m_2}u_2 \\
  \Delta E&=\frac{1}{2}(1-e^2)\frac{m_1m_2}{m_1+m_2}(u_1+u_2)^2
\end{align}

\section{力矩的积分 角动量}
\paragraph{力矩}
\begin{equation}
  \bm{M}=\bm{r}\times\bm{F}
\end{equation}
\paragraph{力偶}
\begin{equation}
  |\bm{M}_{\textrm{duality force}}|=Fd
\end{equation}
\paragraph{角动量}
\begin{equation}
  \bm{L}=\bm{r}\times\bm{p}
\end{equation}
\paragraph{角动量定理}
\begin{equation}
  \bm{M}_{ex}=\frac{dL}{dt}
\end{equation}

\section{质点力学的组合 刚体力学}

\paragraph{角量}
\begin{equation}
  \varphi(t)-\varphi_0=\int_0^t \left( \omega_0 + \int_0^t \alpha dt \right)dt
\end{equation}

\begin{equation}
  \bm{a}=\bm{a}_t+\bm{a}_n=\frac{d\bm{\omega}}{dt}\times \bm{r}+\bm{\omega}\times(\bm{\omega}\times\bm{r})
\end{equation}

\paragraph{转动惯量}
\begin{align}
  J&=\int r^2 dm \\
  M&=J\alpha \\
  L&=J\omega
\end{align}


\begin{center}
  \begin{tabular}{cc}
  \hline
  % after \\: \hline or \cline{col1-col2} \cline{col3-col4} ...
  刚体 & 转动惯量 \\
  \hline
  圆环 & $mR^2$ \\
  圆柱 & $\frac{1}{2}mR^2$ \\
  圆筒 & $\frac{1}{2}m(R_1^2+R_2^2)$ \\
  细棒 & (中部)$\frac{1}{12}ml^2$ \\
  圆球 & $\frac{2}{5}mR^2$ \\
  薄球壳 & $\frac{2}{3}mR^2$ \\
  \hline
\end{tabular}
\end{center}

\paragraph{平行轴定理、正交轴定理}
\begin{align}
  J_A &= J_C+md^2 \\
  J_z &= J_x+J_y
\end{align}

\paragraph{刚体的动能定理}
\begin{equation}
  W_{ex}=\int M_zd\varphi=\frac{1}{2}J\omega^2-\frac{1}{2}J\omega_0^2
\end{equation}

\paragraph{平面平行运动\\}
动能
\begin{equation}
  E_k=\frac{1}{2}J\omega^2
\end{equation}

纯滚动
\begin{align}
  v_C &=R\omega \\
  a_C &=R\alpha
\end{align}

\paragraph{进动}
\begin{equation}
  \bm{M}=\bm{\Omega}\times\bm{L}
\end{equation}

\section{周期性的运动 振动}
\paragraph{简谐振动}
\begin{align}
  m\ddot{x} &=-kx \\
  x&=A\cos(\omega t+\varphi) \\
  \omega &= \sqrt{\frac{k}{m}} \\
  A &= \sqrt{x_0^2+\frac{v_0^2}{\omega^2}} \\
  tan\varphi &= \frac{-v_0}{\omega x_0} \\
  E&=\frac{1}{2}kA^2
\end{align}

\paragraph{谐振子}
\begin{equation}
  \ddot{x}+\omega^2x=0
\end{equation}

\paragraph{振动的合成}
\subparagraph{平行、同频率}

\begin{equation}
  \left\{
    \begin{array}{ll}
      x_1=A_1\cos(\omega t+\varphi_1) \\
      x_2=A_2\cos(\omega t+\varphi_2)
    \end{array}
  \right.
\end{equation}


\begin{align}
  A&=\sqrt{A_1^2+A_2^2+2A_1A_2\cos(\varphi_2-\varphi_1)}\\
    \varphi&=\tan^{-1} \frac{A_1\sin \varphi_1+A_2\sin \varphi_2}{A_1 \cos \varphi_1+A_2 \cos \varphi_2}
\end{align}

\subparagraph{平行、近频率}

 \begin{equation}
  \left\{
    \begin{array}{ll}
      x_1=A\cos(\omega_1 t+\varphi_1) \\
      x_2=A\cos(\omega_2 t+\varphi_2)
    \end{array}
  \right.
 \end{equation}


\begin{align}
  x&=x_1+x_2=2A\cos\frac{\omega_1-\omega_2}{2}t\cos\left(\frac{\omega_1+\omega_2}{2}+\varphi\right) \\
    \Delta\nu&=\left|\frac{\omega_1-\omega_2}{2\pi}\right|
\end{align}

\subparagraph{垂直、同频率}

 \begin{equation}
  \left\{
    \begin{array}{ll}
      x=A_x\cos(\omega t+\varphi_x) \\
      y=A_y\cos(\omega t+\varphi_y)
    \end{array}
  \right.
 \end{equation}


\begin{equation}
  \frac{x^2}{A_x^2}+\frac{y^2}{A_y^2}-\frac{2xy}{A_xA_y}\cos(\varphi_x-\varphi_y)=\sin^2(\varphi_x-\varphi_y)
\end{equation}

\subparagraph{垂直、不同频率\\}
李萨如图形,注意角度起始坐标轴。

\section{超越实体的波}


\paragraph{简谐波}
\begin{align}
  y(x,t)&= A\cos\left[ \omega\left(t-\frac{x}{u}\right)+\varphi \right] \\
  y(x,t)&= A\cos\left(\omega t-kx+\varphi \right) \\
 y(x,t)&= A\cos\left[2\pi \left(\frac{t}{T}-\frac{x}{\lambda}\right)+\varphi\right]
\end{align}

\paragraph{波动方程}
\begin{equation}
  u=\frac{\lambda}{T}=\frac{\omega}{k}
\end{equation}
\begin{equation}
  \frac{\partial^2y}{\partial t^2}=\frac{E}{\rho}\frac{\partial^2y}{\partial x^2}
\end{equation}
\begin{align}
  u_{\parallel}&=\sqrt{\frac{E}{\rho}} \\
  u_{\perp}&=\sqrt{\frac{G}{\rho}}
\end{align}
\begin{align}
  \frac{\partial^2y}{\partial t^2}&=u^2\frac{\partial^2y}{\partial x^2} \\
    u&=\sqrt{\frac{F_T}{\rho_l}}
\end{align}

\paragraph{波的能量与强度}
\begin{align}
  \Delta E_k&=\frac{1}{2}\rho\Delta V \left( \frac{\partial y}{\partial t} \right)^2 \\
  \Delta E_p&=\frac{1}{2}E\Delta V \left( \frac{\partial y}{\partial x} \right)^2
\end{align}

\begin{align}
    \Delta E&=\rho \Delta V \omega^2 A^2 \sin^2\omega \left( t-\frac{x}{u} \right) \\
    \varepsilon &=\frac{\Delta E}{\Delta V}=\rho \omega^2 A^2 \sin^2\omega \left( t-\frac{x}{u} \right) \\
    \bm{I}&=\frac{1}{2} \rho \omega^2 A^2 \bm{u}
\end{align}

\paragraph{球面波}
\begin{equation}
  A\propto \frac{1}{r}
\end{equation}

\paragraph{干涉}
\begin{align}
  y_1 &=A_1\cos(\omega t+\varphi_1-kr_1) \\
  y_2 &=A_2\cos(\omega t+\varphi_2-kr_2) \\
  \Delta &= \varphi_1 - \varphi_2 +k(r_2-r_1)
\end{align}
$\Delta=2n\pi$ 加强;$\Delta=(2n+1)\pi$ 减弱。

\paragraph{驻波}
\begin{align}
  y_1 &=A\cos(\omega t-kx+\varphi_1) \\
  y_2 &=A\cos(\omega t+kx+\varphi_2) \\
  y=y_1+y_2&=2A\cos\left( kx+\frac{\varphi_2-\varphi_1}{2} \right)\cos\left( \omega t+ \frac{\varphi_2+\varphi_1}{2} \right)
\end{align}

$kx+\frac{\varphi_2-\varphi_1}{2}=n\pi$ 波腹; $kx+\frac{\varphi_2-\varphi_1}{2}=\frac{2n+1}{2}\pi$ 波节。

\paragraph{简正模式}
\begin{equation}
  \nu_n=\frac{n}{2l}\sqrt{\frac{F_T}{\rho_l}}
\end{equation}

\paragraph{多普勒效应}
\begin{equation}
  \nu_R=\frac{u+v_R}{u-v_S}\nu_S
\end{equation}
\begin{equation}
  \nu'=\frac{1+\frac{v\cos\theta}{u}}{1-\frac{v\cos\theta}{u}}\nu =\left(1+\frac{2v\cos \theta}{u}\right)\nu (\textrm{if} \quad v\ll u)
\end{equation}
(考虑相对论)
\begin{equation}
  \nu=\sqrt{\frac{c+v}{c-v}} \nu^{\prime}
\end{equation}

\section{光速不变的相对论}
\begin{equation}
  \beta=\sqrt{1-\frac{v^2}{c^2}}
\end{equation}
\paragraph{尺缩、钟慢}
\begin{align}
  l &= l_0 \beta \\
  t &= \frac{t_0}{\beta}
\end{align}

\paragraph{洛伦兹变换\\}
(正变换)
\begin{equation}
\left\{
  \begin{array}{ll}
    x^{\prime}=\frac{x-vt}{\beta} \\
    y^{\prime}=y \\
    z^{\prime}=z \\
    t^{\prime}=\frac{t-\frac{vx}{c^2}}{\beta}
  \end{array}
\right.
\end{equation}

(逆变换)
\begin{equation}
\left\{
  \begin{array}{ll}
    x=\frac{x^{\prime}+vt^{\prime}}{\beta} \\
    y=y^{\prime} \\
    z=z^{\prime} \\
    t=\frac{t^{\prime}+\frac{vx^{\prime}}{c^2}}{\beta}
  \end{array}
\right.
\end{equation}

\paragraph{洛伦兹速度变换\\}
(正变换)
\begin{equation}
\left\{
  \begin{array}{ll}
    u_x^{\prime}=\frac{u_x-v}{1-\frac{vu_x}{c^2}} \\
    u_y^{\prime}=\frac{u_y\beta}{1-\frac{vu_x}{c^2}} \\
    u_z^{\prime}=\frac{u_z\beta}{1-\frac{vu_x}{c^2}}
  \end{array}
\right.
\end{equation}

(逆变换)
\begin{equation}
\left\{
  \begin{array}{ll}
    u_x=\frac{u_x^{\prime}+v}{1+\frac{vu_x^{\prime}}{c^2}} \\
    u_y=\frac{u_y^{\prime}\beta}{1+\frac{vu_x^{\prime}}{c^2}} \\
    u_z=\frac{u_z^{\prime}\beta}{1+\frac{vu_x^{\prime}}{c^2}}
  \end{array}
\right.
\end{equation}

\paragraph{动量与能量}
\begin{align}
  m&=\frac{m_0}{\beta} \\
  E&=mc^2 \\
  \bm{p}&=m\bm{v} \\
  E^2&=p^2c^2+m_0^2c^4
\end{align}

\paragraph{光子}
\begin{equation}
  E=pc
\end{equation}


\end{CJK}

\end{document}
