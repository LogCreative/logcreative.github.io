\documentclass[a4paper]{article}
\usepackage[UTF8]{ctex}
\usepackage{enumitem}
\usepackage{ragged2e}
\usepackage{xcolor}
\usepackage{graphicx}
\usepackage{fontawesome}
\usepackage[colorlinks,urlcolor=blue!30!black]{hyperref}
\usepackage{geometry}
\geometry{left=1.5cm,right=1.5cm,top=1.0cm,bottom=1.0cm}
\newenvironment{cvitems}{
    \begin{justify}
    \begin{description}[
        labelwidth=1.8cm,
        leftmargin=2cm,
        % labelsep=0.5cm,
        parsep=0pt]
    
}{
    \end{description}
    \end{justify}
}
\renewcommand\descriptionlabel[1]{\sffamily #1}

\newenvironment{githubs}{
    \begin{list}{\faGithubSquare}{
        \setlength{\topsep}{0pt}
        \setlength{\partopsep}{0pt}
        \setlength{\parsep}{0pt}
        \setlength{\itemsep}{0pt}}
}{
    \end{list}
}

\newenvironment{subcvitems}{
    \begin{list}{\faCaretSquareORight}{
        \setlength{\topsep}{0pt}
        \setlength{\partopsep}{0pt}
        \setlength{\parsep}{0pt}
        \setlength{\itemsep}{0pt}
        \setlength{\leftmargin}{0.55cm}}
}{
    \end{list}
}
\def\githublink#1{\,\href{https://github.com/LogCreative/#1}{\faGithubSquare{}~\sffamily #1}\,}

\def\section#1{
    \noindent\hskip2cm \textbf{\large #1} \hrulefill
}

\def\rightnote#1{
    {%
    \unskip\nobreak\hfil\penalty50
    \hskip2em\hbox{}\nobreak\hfil\textcolor{gray}{\emph{#1}}%
    \parfillskip=0pt \finalhyphendemerits=0 \par}
}

\def\tip#1{
    {\kaishu #1}~\vrule~
}

\title{\sffamily Log Creative}
\author{\small\sffamily \href{https://github.com/LogCreative}{\faGithub~LogCreative}
\href{https://space.bilibili.com/31271993}{\includegraphics[height=2ex]{bilibili.png}~LogCreative}
\href{mailto:logcreative@outlook.com}{\faEnvelope~logcreative@outlook.com}}
\date{}
\begin{document}
    \begin{minipage}{0.8\textwidth}
        \maketitle
    \end{minipage}
    \begin{minipage}{0.15\textwidth}
        \includegraphics[width=\linewidth]{ProfilePhoto.png}
    \end{minipage}
    
    \thispagestyle{empty}
    
    % 教育经历
    \section{教育经历}
    \begin{cvitems}
        \item[本科生] \textbf{上海交通大学} 三年级 \rightnote{上海}
        
        电子信息与电气工程学院~计算机科学与技术\rightnote{预计 2019--2023}

        数学科学学院~数学与应用数学\rightnote{2018--2019}
    
        \tip{核心学积分} 86.3/100\quad \tip{CET6} 560
        
        \tip{A/A+专业课} 计算机科学中的数学基础、编译原理、计算机系统结构实验。
    \end{cvitems}

    % 技能
    \section{专业能力}
    \begin{cvitems}
        \item[编程语言] 
        \begin{subcvitems}
            \item[\faCheck] 熟悉基于 FLTK 的 \textsf{C/C++}\githublink{GraphGenDecomp}、基于 WinForm/WPF/XAML 的 \textsf{VB{\scriptsize .NET}/C\#\,} \githublink{ClassBookController} \githublink{DialogCreator}、排版程式 \textsf{\LaTeX}\,\githublink{AutoBeamer}相关的界面编程。\rightnote{2015--2021}
            \item[\faCheckCircle] 了解 \textsf{Python} 计算机视觉库、数据分析与机器学习与网络编程\githublink{Networking}。\rightnote{2019--2021}
            \item[\faCheckCircle] 了解 \textsf{HTML/CSS/JavaScript}\,\href{https://logcreative.github.io/LaTeXSparkle/}{\faGithubSquare~\sffamily\LaTeX{}Sparkle} 和 \textsf{Vue.js}\githublink{PGFPlotsEdt}网页制作。\rightnote{2020--2021}
            \item[\faCheckCircleO] 入门 \textsf{Java}\githublink{OS-Projects}、\textsf{Rust}\githublink{CompilerPrinciple}编写小项目。\rightnote{2020}
        \end{subcvitems}
        \item[视觉展现] 
        \begin{subcvitems}
            \item[\faCheck] 熟练地使用 \textsf{Premiere} 视频剪辑、\textsf{After Effects} 制作特效、\textsf{Cinema 4D} 三维建模\,\href{https://www.bilibili.com/video/BV1Qb411a7FH/}{\includegraphics[height=2ex]{bilibili.png}}。\hfill\textcolor{gray}{\emph{2012--2020}}
            \item[\faCheckCircle\hskip1pt] 了解使用 \textsf{Unity} 编写三维交互程序。\textsf{PowerPoint} 讲师。\rightnote{2019--2020}
        \end{subcvitems}
        \item[开源参与] 
        \begin{subcvitems}
            \item[\faCheckCircle] 了解\,\textsf{cmd}/\textsf{PowerShell}/\textsf{bash}\,脚本语言,版本控制工具\,\textsf{Git},持续集成工具\,\textsf{GitHub Actions}\,的使用。
            \item[\faCaretSquareORight\hskip1pt] 校内社群 \href{https://sjtug.org}{\textsf{SJTUG}} 成员,向\TeX{}中文社区 \textsf{C\TeX{}-org} 贡献代码\,\href{https://github.com/CTeX-org/learnlatex.github.io}{\faGithubSquare~\sffamily learnlatex.github.io}。
            \item[\faCaretSquareORight\hskip1.15pt] 参与 \textsf{VS Code} 中文社区线下活动,完成 2021 年微软开源学习社群实践项目\githublink{mnist-calculator} \githublink{custom-tensor-op}\githublink{qlib-CNN}。\rightnote{2021} 
        \end{subcvitems}
    \end{cvitems}

    % 竞赛经历
    \section{获得荣誉}

    \begin{cvitems}
        \item \textbf{美国大学生数学建模竞赛} Meritorious Winner \rightnote{2021}
            \tip{主要贡献} 使用 \textsf{sklearn} 库对 DBSCAN 无监督学习模型的距离重定义,以更好地对森林着火点范围进行预测。使用 Mathematica 进行地理相关的计算。
        \item \textbf{全国大学生物联网竞赛} 全国二等奖 \rightnote{2020}
            \tip{主要贡献} 使用前端技术编写餐厅就餐数据展示网页,将 Nigara 与华为云数据库连接以更好地基于历史数据进行预测。\githublink{SJTUCanteenDataWidget}
        \item \textbf{上海交通大学优秀奖学金} C 等 \rightnote{2018--2019, 2020--2021}
    \end{cvitems}

    % 社团经历
    \section{社团经历}

    \begin{cvitems}
        \item \textbf{上海交通大学 Linux 用户组} $\cdot$ SJTU\TeX{} Maintainer \rightnote{2021}
        \tip{相关工作} \LaTeX{} 幻灯片模板 \href{https://github.com/sjtug/SJTUBeamer}{\faGithubSquare~\sffamily SJTUBeamer} \faStar~\textsf{192}~主要维护者之一。
        \item \textbf{上海交通大学艺术中心} $\cdot$ 技术部部长 \rightnote{2019--2020}
        \tip{相关工作} 视频制作者。编写多路直播管理软件 \href{https://github.com/SJTU-Art-Center/ACLiveConsole}{\faGithubSquare~\sffamily ACLiveConsole} \faStar~\textsf{7}~界面化管理 Nginx,支持对 bilibili 弹幕进行艺术处理与多路视频的导播平滑切换,服务于多场校级晚会与商业活动。
    \end{cvitems}

\end{document}