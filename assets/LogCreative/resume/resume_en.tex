\documentclass[a4paper,11pt]{article}
\usepackage[T1]{fontenc}
\usepackage{mathpazo}
\usepackage{enumitem}
\usepackage{ragged2e}
\usepackage{xcolor}
\usepackage{graphicx}
\usepackage{fontawesome5}
\usepackage[colorlinks,urlcolor=blue!30!black]{hyperref}
\usepackage{geometry}
\geometry{left=1.5cm,right=1.5cm,top=1.0cm,bottom=1.0cm}
\newenvironment{cvitems}{
    \begin{justify}
    \begin{description}[
        % labelwidth=1.8cm,
        leftmargin=0.5cm,
        % labelsep=0.5cm,
        itemindent=-0.5cm,
        parsep=0pt]
    
}{
    \end{description}
    \end{justify}
}
% \renewcommand\descriptionlabel[1]{\hbox to 2.3cm{\sffamily #1}}

\newenvironment{githubs}{
    \begin{list}{\faGithubSquare}{
        \setlength{\topsep}{0pt}
        \setlength{\partopsep}{0pt}
        \setlength{\parsep}{0pt}
        \setlength{\itemsep}{0pt}}
}{
    \end{list}
}

\newenvironment{subcvitems}{
    \begin{list}{{\faCaretSquareRight[regular]\hskip1pt}}{
        \setlength{\topsep}{0pt}
        \setlength{\partopsep}{0pt}
        \setlength{\parsep}{0pt}
        \setlength{\itemsep}{0pt}
        \setlength{\leftmargin}{0.55cm}}
}{
    \end{list}
}
\def\githublink#1{\,\href{https://github.com/LogCreative/#1}{\faGithubSquare{}~\sffamily #1}\,}

\def\section#1{
    %\noindent\hskip2.5cm
    \noindent\textbf{\large\scshape #1} \hrulefill
}

\def\rightnote#1{
    {%
    \unskip\nobreak\hfil\penalty50
    \hskip2em\hbox{}\nobreak\hfil\textcolor{gray}{\emph{#1}}%
    \parfillskip=0pt \finalhyphendemerits=0 \par}
}

\def\tip#1{{\slshape \itshape #1}~\vrule\,}

% decrease the title and author space
\makeatletter
\renewcommand{\maketitle}{
    \begin{center}
        % \vspace{-2.5em} % Adjust this value to change the space above the title
        {\LARGE \@title} \\[3ex]
        {\small \@author} \\[3ex]
        {\small \@date}
    \end{center}
    \vspace{0.1ex} % Adjust this value to change the space below the title
}


\title{\sffamily Log Creative}
\author{\small\sffamily
\href{https://github.com/LogCreative}{\faGithub~LogCreative}~
\href{https://logcreative.tech/about/parts/resume}{\faExternalLink*~logcreative.tech}\hspace*{1.1cm}~~\\
\href{https://space.bilibili.com/31271993}{\includegraphics[height=2ex]{bilibili.png}~LogCreative}~
\href{mailto:logcreative@outlook.com}{\faEnvelope~logcreative@outlook.com}
}
\date{}
\begin{document}
    \begin{minipage}{0.8\textwidth}
        \maketitle
    \end{minipage}
    \begin{minipage}{0.15\textwidth}
        \includegraphics[width=\linewidth]{ProfilePhoto.png}
    \end{minipage}
    
    \thispagestyle{empty}
    
    % 教育经历
    \section{Education Experiences}
    \begin{cvitems}
        \item \textbf{Shanghai Jiao Tong University} Master Student (2nd year) \rightnote{Shanghai, China}
        
        % Electronic Information @ School of Electronic, Information and Electrical Engineering
        Computer Technology @ School of Computer Science
        \rightnote{Expected 2023.9--2026.6} 
        
        \item \textbf{Shanghai Jiao Tong University} Bachelor \rightnote{Shanghai, China}

        Computer Science and Technology @ SEIEE \rightnote{2019.9--2023.6}

        Mathematics @ School of Mathematical Sciences \rightnote{Transferred 2018.9--2019.8}
    
        %\tip{GPA} 3.71/4.3\quad \tip{CET6} 545 (Oral: B+)
        
        % \tip{A/A+ Courses} VR, Machine Learning, Data Visualization, Mathematical Foundamental in Computer Sciences, Compiler Principles, Computer Architechture Labs
    \end{cvitems}

    % 技能
    \section{Skills}
    \begin{cvitems}
        % \item[Programming] 
        % \begin{subcvitems}
            \item \faCheck~Familiar with \textsf{Python} (\textsf{pytorch}, \textsf{transformers}, ...), \textsf{C/C++}, \textsf{C\#/VB{\scriptsize .NET}}, \textsf{\LaTeX} programming. \rightnote{2018--2025}
            \item \faCheckCircle~Acknowledge \textsf{HTML/CSS/JavaScript}, \textsf{Vue.js} and \textsf{Next.js} frontend framework, an open-source full-stack project: \githublink{PGFPlotsEdt} \faStar~\textsf{74}, an openInula class project: \href{https://gitee.com/LogCreative/inula-docdemo}{\textsf{inula-docdemo}}. \rightnote{2020--2025}
            \item {\faCaretSquareRight[regular]}\hskip1.5pt~Get started with \textsf{Java} and \textsf{Rust}. \rightnote{2020}
            \item \faCheckCircle~Acknowledge \,\textsf{cmd}/\textsf{PowerShell}/\textsf{bash}/\textsf{lua}\,script languages, version control tool\,\textsf{Git}, and continous integration tool\,\textsf{GitHub Actions} for constructing workflows of various open source projects. \rightnote{2021--2025}
            % \item[\faCheck] Familiar with interface programming like FLTK in \textsf{C/C++}\githublink{GraphGenDecomp}, WinForm/WPF/XAML in \textsf{VB{\scriptsize .NET}/C\#\,} \githublink{ClassBookController} \githublink{DialogCreator}, Typesetting of \textsf{\LaTeX}\,\githublink{AutoBeamer}.\rightnote{2015--2021}
            % \item[\faCheckCircle] Acknowledge OpenCV, pandas, pytorch and networking \githublink{Networking} in \textsf{Python}.\rightnote{2019--2021}
            % \item[\faCheckCircle] Acknowledge webpage making in \textsf{HTML/CSS/JavaScript}\,\href{https://logcreative.github.io/LaTeXSparkle/}{\faGithubSquare~\sffamily\LaTeX{}Sparkle} (Blog), and use JavaScript framework \textsf{Vue.js}\githublink{PGFPlotsEdt}.\rightnote{2020--2021}
            % \item[{\faCheckCircle[regular]}] Get started with \textsf{Java}\githublink{OS-Projects} and \textsf{Rust}\githublink{CompilerPrinciple}.\rightnote{2020}
        % \end{subcvitems}
        % \item[Visualization] 
        % \begin{subcvitems}
            % \item[\faCheck] Familiar with Video Editing in \textsf{Premiere}, Special Effects in \textsf{After Effects}, 3D Modeling in \textsf{Cinema 4D}\,\href{https://www.bilibili.com/video/BV1Qb411a7FH/}{\includegraphics[height=2ex]{bilibili.png}}.\hfill\textcolor{gray}{\emph{2012--2020}}
            \item \faCheckCircle\hskip1pt~Acknowledge \textsf{Unity} to make Virtual Reality (VR) programs. Previously researched in VR full body gesture reconstruction and collision sound simulation. %\textsf{PowerPoint} Lecturer.
            \rightnote{2019--2020, 2023--2024}
        % \end{subcvitems}
        % \item[Open Source] 
        % \begin{subcvitems}
        % \end{subcvitems}
    \end{cvitems}

    % 社团经历
    \section{Internship \& Community}

    \begin{cvitems}
        \item \textbf{Shanghai Baosight Software Co., Ltd.} Intern \rightnote{2024--Present}
        
        Research on LLM and Agent, focusing on LLM routing and Text-to-SQL problems. Use SFT and GRPO algorithms to finetune the LLM based on \textsf{transformers} library, and construct the agent framework based on \textsf{LangChain}/\textsf{LangGraph} to improve the automation level of many tasks.

        \item \textbf{SJTU Linux User Group} $\cdot$ SJTU\TeX{} Maintainer \rightnote{2021--Present}
        Maintainer of \LaTeX{} Beamer Template \href{https://github.com/sjtug/SJTUBeamer}{\faGithubSquare~\sffamily SJTUBeamer} \faStar~\textsf{717}, and one of the authors of \LaTeX{} lecture \href{https://github.com/sjtug/sjtulib-latex-talk}{\faGithubSquare~\sffamily sjtulib-latex-talk} \faStar~\textsf{67}, servicing the \TeX{} users across the campus.
        \item \textbf{SJTU Art Center} $\cdot$ Leader of Tech Department \rightnote{2019--2020}
        \tip{Related Work} Video Producer. Create multi-streaming software \href{https://github.com/SJTU-Art-Center/ACLiveConsole}{\faGithubSquare~\sffamily ACLiveConsole} \faStar~\textsf{7}~to manage nginx visually, support art processing of bilibli danmaku, and fluent PGM animation. The software has serviced 12 in-school events, including the commercial ones.
    \end{cvitems}

    % 竞赛经历
    \section{Contests}

    \begin{cvitems}
        \item \textbf{National Undergraduate IOT Design Contest} $\cdot$ HarmonyOS Special Award (10/1417) \rightnote{2022}
            Develop a HarmonyOS App based on the technology of distributed database, to service the hybrid class with social distancing. Build a cloud platform to interconnect clients, \href{http://iot.logcreative.tech}{web}, and a hardware.
        \item \textbf{Microsoft Research Asia Practice Space Project} \rightnote{2021}
            \githublink{mnist-calculator} creates a hand-written calculator based on CNN, \githublink{custom-tensor-op} implements convolution layer based on \textsf{pytorch} API (Python/C++), \githublink{qlib-CNN} implements a simple convolution model to make analysis on stock market.
        \item \textbf{Mathematical Contest in Modeling (MCM)} $\cdot$ Meritorious Winner \rightnote{2021}
            Redefine the distance of DBSCAN in Python \textsf{sklearn} library, in order to perform a better prediction in the range of wildfire. Perform geographical calculations in Mathematica.
    \end{cvitems}

    \section{Publication}
    \begin{cvitems}
        \item \href{https://ieeexplore.ieee.org/abstract/document/10361019}{\textbf{ViTframe: Vision Transformer Acceleration via Informative Frame Selection for Video Recognition.}} In: \textit{IEEE International Conference on Computer Design (ICCD)}. Second author. \rightnote{2023}
        In this paper, we propose a novel method to accelerate the inference of video recognition models like TimeSformer and MViT. We introduce a frame selection mechanism that selects informative frames based on the metric of SSIM from the video input, reducing the computational cost while maintaining accuracy loss within 1\%.
    \end{cvitems}

    % \section{Honors}
    % \begin{cvitems}
    %     \item \textbf{SJTU Huawei Scholarship} (10/139) \rightnote{2024}
    %     \item \textbf{SJTU CS Outstanding Bachelor's Thesis} (20/161) \rightnote{2023}
    % \end{cvitems}

    

\end{document}